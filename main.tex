\documentclass[a4paper,capitulo,12pt]{br-lex}
\usepackage[top=2.5cm,bottom=2.5cm,left=2.5cm,right=2.5cm]{geometry}
\usepackage{lmodern}
\usepackage{libertine}
\exhyphenpenalty=1000
\hyphenpenalty=1000
\widowpenalty=1000
\clubpenalty=1000 % evitar separação de sílabas


%%%%% Comentários sobre o uso do pacote:

% Seções são inseridas com \section{nome} 
% Subseções com \subsection{nome}
% Não esquecer de pular linhas entre os parágrafos
% De resto, os demais comandos estão descritos no texto abaixo.

\begin{document}

\begin{center}
    Presidência da República\\
    Casa Civil\\
    Subchefia para Assuntos Jurídicos   
\end{center}

\titulo{Lei Complementar Nº 95,\\ 
       de 26 de fevereiro de 1998}

\descricao{Afirma que XXXXXXX e dá outras providências.}

O PRESIDENTE DA REPÚBLICA Faço saber que o Congresso  Nacional decreta
e eu sanciono a seguinte Lei Complementar:

\section{PRIMEIRA SEÇÃO}

\chapter{DISPOSIÇÕES PRELIMINARES}
\label{chap:disposicoes}

\artigo A elaboração, a redação, a alteração e a consolidação das leis
obedecerão ao disposto nesta Lei Complementar.

Parágrafo único. As disposições desta Lei Complementar aplicam-se, ainda, às medidas provisórias e demais atos normativos referidos no art. 59 da Constituição Federal, bem como, no que couber, aos decretos e aos demais atos de regulamentação expedidos por órgãos do Poder Executivo.

\artigo \cortado{Este artigo foi vetado}

\artigo Artigo de exemplo.

\begin{paragrafos}

\paragrafo Primeiro

\paragrafo Segundo parágrafo

\end{paragrafos}

\begin{easylist}

# Teste

## Nível abaixo

### Mais um subnível

\end{easylist}

\end{document}